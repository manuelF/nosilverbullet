\documentclass{beamer}
\usepackage{etex}
\usepackage[utf8]{inputenc}
\usepackage{xcolor}
\usepackage{xspace}
\usepackage[all]{xy}
\usepackage{amsmath,amssymb,amsthm}
\usepackage{stmaryrd}
%\usepackage{qtree}
%\usepackage{hyperref}
\usepackage{listings}
\usepackage{mathrsfs}

\usepackage[spanish]{babel}
\uselanguage{Spanish}
\languagepath{Spanish}

\AtBeginDocument{\shorthandoff{"}}

%%%% Inductive rules with prooftree
%\input{prooftree.tex}
%\newcommand{\indrulename}[1]{\textsc{#1}}
%\newcommand{\indrule}[3]{
%\ensuremath{
%\begin{array}{c}
%  \prooftree #2
%    \justifies #3
%    \thickness=0.05em
%    \using \indrulename{#1}
%  \endprooftree
%\end{array}}}

%\beamertemplatenavigationsymbolsempty

\definecolor{color_titulo}{rgb}{0.25,.45,.25}
%\newcommand{\J}[1]{\textcolor{blue}{#1}}
%\newcommand{\HS}{\hspace{.5cm}}
%\newcommand{\Bool}{\mathbb{B}}
%\newcommand{\fv}[1]{\textsf{fv}(#1)}
%\newcommand{\Nat}{\mathbb{N}}
%\newcommand{\refl}{\textsf{refl}}
%\newcommand{\ap}{\textsf{ap}}
%\newcommand{\apd}{\textsf{apd}}
%\newcommand{\lift}{\textsf{lift}}
%\newcommand{\tr}{\textsf{tr}}
%\newcommand{\id}{\textsf{id}}
%\newcommand{\ind}{\textsf{ind}}
%\newcommand{\rec}{\textsf{rec}}
%\newcommand{\U}{\mathcal{U}}
%\newcommand{\ctx}{\,\textsf{ctx}}
%%\newcommand{\eq}[1]{\,=\raisebox{-.1cm}{\ensuremath{_{\!\!#1}}}\,}
%\newcommand{\eq}[1]{\,=_{#1}\,}
%\newcommand{\A}[1]{\Pi_{(#1)}}
%\newcommand{\E}[1]{\Sigma_{(#1)}}
%\newcommand{\whil}{\bullet_{\ell}}
%\newcommand{\whir}{\bullet_{r}}

\begin{document}
{
%\usebackgroundtemplate{\includegraphics[width=12.8cm]{hott.png}}
\frame[plain]{
\vspace{4cm}
\begin{center}
 {\LARGE \textcolor{color_titulo}{{\em Homotopy Type Theory}}}\\
 \vspace{.25cm}
 {\large Tercer encuentro -- 18 de septiembre de 2014}\\
 \vspace{.5cm}
 {\bf {Espacio sobre temas de programaci\'on funcional}}\\
 \vspace{.5cm}
 {\small
    Facultad de Ciencias Exactas y Naturales\\
    Universidad de Buenos Aires \\
    2014 }
\end{center}
}
}

\begin{frame}[fragile]
  \frametitle{Qui\'en es \textit{Frederick P. Brooks, Jr.}}
\end{frame}

\begin{frame}[fragile]
  \frametitle{Introducci\'on}
\end{frame}


\begin{frame}[fragile]
  \frametitle{Dificultades esenciales}

  A diferencia del mundo del \textit{hardware} no podemos esperar que nuestra
  capacidad de producir buen \textit{sofware} se doble cada a\~no.

  De hecho, deber\'iamos considerar el caso del \textit{hardware} como una
  excepci\'on.

  La esencia de una entidad de \textit{software} es el resultado de la
  composici\'on no trivial de varios conceptos:
  \begin{itemize}
    \item conjuntos de datos
    \item relaciones entre datos
    \item algoritmos
    \item invocaci\'on de funciones
  \end{itemize}

  La parte dif\'icil de construir \textit{software} es la
  \textbf{especificaci\'on}, \textbf{dise\~no} y \textbf{testeo} de estos
  conceptos.
  La dificultad \textbf{no se encuentra} en la representaci\'on y el testeo
  de la fidelidad de la representaci\'on.

  \textbf{Los errores sint\'acticos son muy menores en comparaci\'on a los
  errores conceptuales en la mayor\'ia de los sistemas.}
\end{frame}


\begin{frame}[fragile]
  \frametitle{Complejidad}

  Las entidades de \textit{software} son complejas porque dos partes en un
  \textit{software} nunca son semejantes, si as\'i lo fuera, uno deber\'ia
  abstraerlas en una misma subrutina. Por este motivo el \textit{software}
  se diferencia de las computadoras, los edificios o los autom\'obiles, donde
  los elementos repetidos abundan.

  Lograr que una entidad de \textit{software} escale es complejo, pues los
  elementos interactuan de forma no lineal.

  La complejidad del \textit{software} es esencial, no accidental. Por lo tanto
  descripciones que abstraigan esta complejidad suelen abstraer la esencia.

  La complejidad genera:
  \begin{itemize}
    \item dificultad en la comunicaci\'on de un equipo
      $\mapsto$ fallas, costo, atraso del proyecto
    \item dificultad en la enumeraci\'on de los estados de un programa
      $\mapsto$ baja confiabilidad en el programa
    \item complejidad en las funciones 
      $\mapsto$ dificultad en el uso de las mismas
    \item complejidad en las estructuras usadas
      $\mapsto$ dificultades en la modificabilidad, problemas de seguridad
  \end{itemize}

\end{frame}


\begin{frame}[fragile]
  \frametitle{\textit{Conformity} (adaptaci\'on)}

  La f\'isica, y otras disciplinas, deben lidiar con objetos complejos, pero
  se basan en la expectativa de encontrar teor\'ias unificadas.
  Esta expectativa no conforta a la ingenier\'ia del \textit{software}.

  La complejidad con la que se encuentra una entidad de \textit{software}
  es arbitraria, generada por las muchas instituciones humanas y sistemas
  a los cuales, sus interfaces, deben acatar.

  En los (muchos) casos en que una entidad de \textit{software} debe adaptarse
  a otras interfaces, la complejidad que esto genera no puede ser simplificada
  por ning\'un redise\~no que incluya \'unicamente a dicha entidad.
\end{frame}


\begin{frame}[fragile]
  \frametitle{Modificabilidad}
  A diferencia de los objetos manufacturados \textit{f\'isicamente} (edificios,
  autom\'obiles o computadoras), las entidades de \textit{software} continuamente
  est\'an siendo forzadas a cambiar.

  Los motivos principales son dos:
  \begin{itemize}
    \item la funcionalidad de un sistema suele necesitar extenderse o modificarse
    \item a diferencia de objetos f\'isicamente concretos, el \textit{software}
      es infinitamente modificable
  \end{itemize}

  Un sistema exitoso sufrir\'a modificaciones. Mientras se lo encuentre \'util,
  los usuarios lo utilizar\'an en escenarios no contemplados, en otros dominios.
  M\'as a\'un, un sistema exitoso sobrevivir\'a a las maquinas para las cuales
  fue inicialmente construido, que, sabemos, cambian continuamente.
\end{frame}


\begin{frame}[fragile]
  \frametitle{Invisibilidad}
  El \textit{software} es invisibile, \textit{invisualizable}.
  Las abstracciones geom\'etricas suelen resaltar inconsistencias, pero
  la realidad del \textit{software} no est\'a naturalmente embebida
  en un espacio.

  Las herramientas m\'as comunes para la visualizaci\'on suelen ser poco claras,
  pues vemos demasiadas conexiones. Suele ser buena pr\'actica limitar los
  \textit{links} y preferir los grafos \textbf{jer\'arquicos}.

  La visualizaci\'on geom\'etrica es una de las herramientas conceptuales m\'as
  importantes de la mente humana. La dificultad de visualizar el \textit{software}
  no solo limita el dise\~no de una mente, \textbf{limita la comunicaci\'on entre
  varias mentes}.
\end{frame}


\begin{frame}[fragile]
  \frametitle{Dificultades accidentales que fueron resueltas}

  \begin{itemize}
    \item[Lenguajes de alto nivel]
      Liberaron a los programas de mucha de su complejidad accidental.
      Nuesta sofisticaci\'on en c\'omo pensar las estructuras de datos, los tipos
      de datos y las operaciones, se encuentra en crecimiento, pero es un crecimiento
      cada vez m\'as limitado.
    \item[Tiempo compartido]
      El efecto principal es la r\'apida respuesta de un sistema. A medida que
      tiende a cero, el \textit{delay} se torna imperceptible. Mas all\'a de
      este punto no pueden esperarse beneficios particularmente importantes.
    \item[



\end{frame}

%\begin{frame}[fragile]
%\frametitle{Estructura de $\infty$-groupoid}
%En principio, el \'unico constructor de la igualdad es la reflexividad:
%$$\J{\refl : \A{a:A}\,{(a \eq{A} a)}}$$
%El tipo $\J{x \eq{A} y}$ trae aparejado el principio de inducci\'on en caminos:
%$$
%  \J{
%    J :
%      \begin{array}[t]{rl}
%          \A{C : \A{x,y:A}{\,(x \eq{A} y)\,\to\,\U}} \\
%              & (\A{x:A}{C(x,x,\refl_x)}) \\
%          \to & \A{x,y:A}{\A{p : x \eq{A} y}\,C(x,y,p)}
%      \end{array}
%  }
%$$
%que cumple con la siguiente regla de c\'omputo:
%$$
%  \J{
%    J(C, f, x, x, \refl_x) :\equiv f(x)
%  }
%$$
%\end{frame}
%
%\begin{frame}[fragile]
%\frametitle{Estructura de $\infty$-groupoid}
%Una lectura informal del principio de inducci\'on en caminos:
%  $$
%    \J{
%      J :
%        \begin{array}[t]{rl}
%            \underbrace{\A{C : \A{x,y:A}{\,(x \eq{A} y)\,\to\,\U}}}_{(1)} \\
%                & \underbrace{(\A{x:A}{C(x,x,\refl_x)})}_{(2)} \\
%            \to & \underbrace{\A{x,y:A}{\A{p : x \eq{A} y}\,C(x,y,p)}}_{(3)}
%        \end{array}
%    }
%  $$
%
%\begin{itemize}
%\item[(1)]
%  Dada cualquier propiedad $\J{C}$ que
%  dependa de un par de puntos $\J{x, y : A}$ y
%  un camino $\J{p : x \eq{A} y}$ entre ellos,
%\item[(3)]
%  si se quiere construir una demostraci\'on de
%  $\J{C(x,y,p)}$ para todo par de puntos
%  $\J{x, y : A}$ y cualquier camino $\J{p : x \eq{A} y}$,
%\item[(2)]
%  basta construir una demostraci\'on de $\J{C(x,x,\refl_x)}$
%  para cada punto $\J{x : A}$.
%\end{itemize}
%\end{frame}
%
%\begin{frame}[fragile]
%\frametitle{Estructura de $\infty$-groupoid}
%El principio $\J{J}$ de inducci\'on en caminos es
%relativamente complejo, pero permite concluir
%todas las propiedades deseables de la igualdad:
%
%\begin{itemize}
%\item[] {\bf Lema} (simetr\'ia){\bf .}
%  Para cada tipo $\J{A : \U}$ y elementos $\J{x, y: A}$
%  existe una funci\'on
%  $$
%    \J{(x \eq{A} y) \to (y \eq{A} x)}
%  $$
%  notada $\J{p \mapsto p^{-1}}$ tal que
%  $\J{\refl_x^{-1} \equiv \refl_x}$ para cada $\J{x : A}$.
%  Llamamos a $\J{p^{-1}}$ la {\em inversa} de $\J{p}$.
%\end{itemize}
%
%\textcolor{red}{{\em Hacer las dos versiones de la demostraci\'on.}}
%\end{frame}
%
%\begin{frame}[fragile]
%\frametitle{Estructura de $\infty$-groupoid}
%\begin{itemize}
%\item[] {\bf Lema} (transitividad){\bf .}
%  Para cada tipo $\J{A : \U}$ y elementos $\J{x, y, z: A}$
%  existe una funci\'on
%  $$
%    \J{(x \eq{A} y) \to (y \eq{A} z) \to (x \eq{A} z)}
%  $$
%  notada $\J{p \mapsto q \mapsto p \cdot q}$ tal que
%  $\J{\refl_x \cdot \refl_x \equiv \refl_x}$
%  para cada $\J{x : A}$.
%\end{itemize}
%
%\textcolor{red}{{\em Hacer las dos versiones de la demostraci\'on.}}
%\end{frame}
%
%\begin{frame}[fragile]
%\frametitle{Estructura de $\infty$-groupoid}
%\begin{itemize}
%\item[] {\bf Lema} (coherencia de las operaciones){\bf .}
%  Sean un tipo $\J{A : \U}$,
%  puntos $\J{x, y, z, w : A}$
%  y caminos
%  $$\J{p : x \eq{A} y} \HS \J{q : y \eq{A} z} \HS \J{r : z \eq{A} w}$$
%  Se tiene:
%  \begin{enumerate}
%  \item[\textcolor{black}{1.}] $\J{p = p \cdot \refl_y}$ \quad y \quad $\J{p = \refl_x \cdot p}$
%  \item[\textcolor{black}{2.}] $\J{p^{-1} \cdot p = \refl_y}$ \quad y \quad $\J{p \cdot p^{-1} = \refl_x}$
%  \item[\textcolor{black}{3.}] $\J{(p^{-1})^{-1} = p}$
%  \item[\textcolor{black}{4.}] $\J{p \cdot (q \cdot r) = (p \cdot q) \cdot r}$
%  \end{enumerate}
%  Son igualdades proposicionales, o caminos entre caminos.
%  Por ejemplo, el \'item 4. en realidad se lee:
%  $$\J{p \cdot (q \cdot r) \eq{(x \eq{A} w)} (p \cdot q) \cdot r}$$
%\end{itemize}
%\end{frame}
%
%\begin{frame}[fragile]
%\frametitle{Teorema de Eckmann-Hilton}
%
%Si $\J{A : \U}$ es un tipo y $\J{a : A}$ un punto,
%se define el espacio de lazos como el tipo de los
%caminos entre $\J{a}$ y $\J{a}$:
%
%$$\J{\Omega(A,a) : \U}$$
%$$\J{\Omega(A,a) :\equiv a \eq{A} a}$$
%
%Notar que $\J{\refl_a : \Omega(A,a) \equiv a \eq{A} a}$.
%Se define el espacio de lazos 2-dimensionales como el espacio
%de lazos de $\J{\refl_a}$:
%
%$$\J{\Omega^2(A,a) : \U}$$
%$$
%\begin{array}{rrl}
%\J{\Omega^2(A,a)} & \J{:\equiv} & \J{\Omega(a \eq{A} a, \refl_a)} \\
%                  & \J{\equiv} & \J{\refl_a \eq{(a \eq{A} a)} \refl_a} \\
%\end{array}
%$$
%Fijado un punto $\J{a : A}$, escribimos $\J{\Omega(A)}$ por $\J{\Omega(A,a)}$.
%\end{frame}
%
%\begin{frame}[fragile]
%\frametitle{Teorema de Eckmann-Hilton}
%\begin{itemize}
%\item[] {\bf Teorema} (Eckmann-Hilton){\bf.}
%La operaci\'on de composici\'on del espacio de lazos 2-dimensionales:
%  $$
%    \J{\Omega^2(A) \times \Omega^2(A) \to \Omega^2(A)}
%  $$
%es conmutativa:
%  $$
%    \J{\alpha\cdot\beta = \beta\cdot\alpha}
%    \text{ para cada $\J{\alpha, \beta : \Omega^2(A)}$.}
%  $$
%\end{itemize}
%\end{frame}
%
%\begin{frame}[fragile]
%\frametitle{Teorema de Eckmann-Hilton}
%\begin{itemize}
%\item
%Para demostrar Eckmann-Hilton,
%considerar la siguiente situaci\'on m\'as general:
%$$
%\xymatrix@R=.5cm{
%&
%  \ar@{}[dd]^(.25){}="a"^(.75){}="b" \ar@{=>}"a";"b"^{\alpha}
%&
%&
%  \ar@{}[dd]^(.25){}="a"^(.75){}="b" \ar@{=>}"a";"b"^{\beta}
%&
%\\
%x
%\ar@/^{.5cm}/[rr]^{p_1}
%\ar@/_{.5cm}/[rr]_{p_2}
%&
%&
%y
%\ar@/^{.5cm}/[rr]^{q_1}
%\ar@/_{.5cm}/[rr]_{q_2}
%&
%&
%z
%\\
%& & & &
%}
%$$
%\item Definir las operaciones de {\em whiskering}:
%  $$\J{\whil : \A{p : x = y}\ \A{\beta : q_1 = q_2}\ p \cdot q_1 = p \cdot q_2}$$
%  $$\J{\whir : \A{\alpha : p_1 = p_2}\ \A{q : y = z}\ p_1 \cdot q = p_2 \cdot q}$$
%  usando que ya probamos:
%  $$
%  \begin{array}{l|l}
%  \J{u : \A{p:x=y}(p = \refl_x \cdot p)}
%  &
%  \J{v : \A{p:x=y}(p = p \cdot \refl_y)}
%  \\
%  \J{u(\refl_x) \equiv \refl_{\refl_x}}
%  &
%  \J{v(\refl_x) \equiv \refl_{\refl_x}}
%  \end{array}
%  $$
%\end{itemize}
%\end{frame}
%
%\begin{frame}[fragile]
%\frametitle{Teorema de Eckmann-Hilton}
%\begin{itemize}
%\item Por c\'omo est\'an definidas, las operaciones de {\em whiskering} cumplen:
%  $$\J{\refl_x \whil \beta : \refl_x \cdot q_1 = \refl_x \cdot q_2}$$
%  $$\J{\refl_x \whil \beta \equiv u(q_1)^{-1} \cdot \beta \cdot u(q_2)}$$
%
%  $$\J{\alpha \whir \refl_y : p_1 \cdot \refl_y = p_2 \cdot \refl_y}$$
%  $$\J{\alpha \whir \refl_y \equiv v(p_1)^{-1} \cdot \alpha \cdot v(p_2)}$$
%\end{itemize}
%\end{frame}
%
%\begin{frame}[fragile]
%\frametitle{Teorema de Eckmann-Hilton}
%\begin{itemize}
%\item Definir la composici\'on horizontal de dos maneras:
%\end{itemize}
%$$\J{\alpha \star \beta : p_1 \cdot q_1 = p_2 \cdot q_2}$$
%$$\J{\alpha \star \beta :\equiv
%  \underbrace{(\alpha \whir q_1)}_{p_1 \cdot q_1 = p_2 \cdot q_1}
%  \cdot
%  \underbrace{(p_2 \whil \beta)}_{p_2 \cdot q_1 = p_2 \cdot q_2}
%}$$
%
%$$
%\begin{array}{lllll}
%  \xymatrix@R=.2cm@C=.2cm{
%  \ar@/^{.3cm}/[rr]^{} & & \ar@/^{.3cm}/[rr]^{} & & \\
%  }
%&
%  \Rightarrow
%&
%  \xymatrix@R=.2cm@C=.2cm{
%  \ar@/_{.3cm}/[rr]_{} & & \ar@/^{.3cm}/[rr]^{} & & \\
%  }
%&
%  \Rightarrow
%&
%  \xymatrix@R=.2cm@C=.2cm{
%  \ar@/_{.3cm}/[rr]_{} & & \ar@/_{.3cm}/[rr]_{} & & \\
%  }
%\end{array}
%$$
%
%$$\J{\alpha \star' \beta : p_1 \cdot q_1 = p_2 \cdot q_2}$$
%$$\J{\alpha \star' \beta :\equiv
%  \underbrace{(p_1 \whil \beta)}_{p_1 \cdot q_1 = p_1 \cdot q_2}
%  \cdot
%  \underbrace{(\alpha \whir q_2)}_{p_1 \cdot q_2 = p_2 \cdot q_2}
%}$$
%
%$$
%\begin{array}{lllll}
%  \xymatrix@R=.2cm@C=.2cm{
%  \ar@/^{.3cm}/[rr]^{} & & \ar@/^{.3cm}/[rr]^{} & & \\
%  }
%&
%  \Rightarrow
%&
%  \xymatrix@R=.2cm@C=.2cm{
%  \ar@/^{.3cm}/[rr]^{} & & \ar@/_{.3cm}/[rr]_{} & & \\
%  }
%&
%  \Rightarrow
%&
%  \xymatrix@R=.2cm@C=.2cm{
%  \ar@/_{.3cm}/[rr]_{} & & \ar@/_{.3cm}/[rr]_{} & & \\
%  }
%\end{array}
%$$
%\end{frame}
%
%\begin{frame}[fragile]
%\frametitle{Teorema de Eckmann-Hilton}
%\begin{itemize}
%\item {\bf Afirmaci\'on.}
%  Las composiciones horizontales $\J{\star}$ y $\J{\star'}$ coinciden:
%  $$\J{\A{\alpha : p_1 = p_2}\A{\beta : q_1 = q_2}\ \alpha \star \beta = \alpha \star' \beta}$$
%  {\em Demostraci\'on.} Por inducci\'on en $\J{\alpha}$ y $\J{\beta}$,
%  asumiendo
%  $$\J{p :\equiv p_1 \equiv p_2} \quad \J{\alpha \equiv \refl_p}$$
%  $$\J{q :\equiv q_1 \equiv q_2} \quad \J{\beta \equiv \refl_q}$$
%  basta probar
%  $$
%    \J{
%      \refl_p \star \refl_q
%      \eq{(p \cdot q = p \cdot q)}
%      \refl_p \star' \refl_q
%    }.
%  $$
%  Por inducci\'on en $\J{p}$ y $\J{q}$, asumiendo
%  $$\J{x \equiv y \equiv z} \quad \J{p \equiv q \equiv \refl_x}$$
%  basta probar
%  $$
%    \J{
%      \refl_{\refl_x} \star \refl_{\refl_x}
%      \eq{(\refl_x = \refl_x)}
%      \refl_{\refl_x} \star' \refl_{\refl_x}
%    }
%  $$
%  Cuentas. $\Box$
%\end{itemize}
%\end{frame}
%
%\begin{frame}[fragile]
%\frametitle{Teorema de Eckmann-Hilton}
%\begin{itemize}
%\item En el caso de $\J{\Omega^2(A)}$, la situaci\'on es:
%$$\J{x \equiv y \equiv z}$$
%$$\J{p_1 \equiv p_2 \equiv q_1 \equiv q_2 \equiv \refl_x}$$
%Dados lazos 2-dimensionales $\J{\alpha, \beta : \Omega^2(A)}$ se tiene:
%$$
%  \begin{array}{cll}
%              & \J{\alpha \star \beta} \\
%  \J{\equiv}  & \J{(\alpha \whir q_1) \cdot (p_2 \whil \beta)}         \\
%  \J{\equiv}  & \J{(\alpha \whir \refl_x) \cdot (\refl_x \whil \beta)} \\
%  \J{\equiv}  & \J{v(\refl_x)^{-1} \cdot \alpha \cdot v(\refl_x) \cdot u(\refl_x)^{-1} \cdot \beta \cdot u(\refl_x)} \\
%  \J{\equiv}  & \J{\alpha \cdot \beta} \\
%  \end{array}
%$$
%An\'alogamente, $\J{\alpha \star' \beta \equiv \beta \cdot \alpha}$.
%
%Para terminar, $\J{\alpha \cdot \beta \equiv \alpha \star \beta \underbrace{=}_{\text{\textcolor{black}{afirmaci\'on}}} \alpha \star' \beta \equiv \beta \cdot \alpha}$.
%\end{itemize}
%\end{frame}
%
%\begin{frame}[fragile]
%\frametitle{Las funciones son funtores}
%\begin{itemize}
%\item[] {\bf Lema.} Dadas una funci\'on $\J{f : A \to B}$ y
%      puntos $\J{x, y : A}$, hay una funci\'on:
%      $$
%      \J{\ap_f : (x \eq{A} y) \to (f(x) \eq{B} f(y))}
%      $$
%      tal que $\J{\ap_f(\refl_x) \equiv \refl_{f(x)}}$ para cada $\J{x : A}$.
%\end{itemize}
%
%A veces se nota $\J{f(p)}$ para $\J{\ap_f(p)}$.
%\end{frame}
%
%\begin{frame}[fragile]
%\frametitle{Las funciones son funtores}
%\begin{itemize}
%\item[] {\bf Lema.}
%  Sean funciones $\J{f : A \to B}$, $\J{g : B \to C}$
%  y caminos $\J{p : x \eq{A} y}$, $\J{q : y \eq{A} z}$.
%  Entonces:
%  \begin{enumerate}
%  \item[\textcolor{black}{1.}] $\J{\ap_f(p \cdot q) = \ap_f(p) \cdot \ap_f(q)}$
%  \item[\textcolor{black}{2.}] $\J{\ap_f(p^{-1}) = \ap_f(p)^{-1}}$
%  \item[\textcolor{black}{3.}] $\J{\ap_g(\ap_f(p)) = \ap_{g \circ f}(p)}$
%  \item[\textcolor{black}{4.}] $\J{\ap_{\id_A}(p) = p}$
%  \end{enumerate}
%\end{itemize}
%\end{frame}
%
%\begin{frame}[fragile]
%\frametitle{Las funciones son funtores}
%?`C\'omo se generalizar\'ia para funciones dependientes?
%$$\J{f : \A{x:A}{B(x)}}$$
%$$\J{p : x \eq{A} y}$$
%$$\J{f(x) : B(x)}$$
%$$\J{f(y) : B(y)}$$
%{\em A priori}, $\J{f(x)}$ y $\J{f(y)}$ son incomparables.
%\end{frame}
%
%\begin{frame}[fragile]
%\frametitle{Las familias son fibraciones}
%\begin{itemize}
%\item[] {\bf Lema} (transporte){\bf.} Si $\J{P : A \to \U}$ es una familia de tipos,
%      y $\J{p : x \eq{A} y}$, hay una funci\'on:
%      $$
%        \J{\tr^P(p) : P(x) \to P(y)}
%      $$
%\end{itemize}
%
%Observaciones:
%\begin{itemize}
%\item Cumple la propiedad $\J{\tr^P(\refl_x) \equiv id_{P(x)}}$.
%
%\item Si la familia $\J{P}$ est\'a clara en el contexto,
%      se nota $$\J{p^* : P(x) \to P(y)}$$
%
%\item Si se piensa $\J{P}$ como un predicado, el lema de
%      transporte afirma que respeta la igualdad:
%      toda evidencia de $\J{P(x)}$ se transporta a evidencia de $\J{P(y)}$.
%\end{itemize}
%\end{frame}
%
%\begin{frame}[fragile]
%\frametitle{Las familias son fibraciones}
%Interpretaci\'on topol\'ogica del lema de transporte\footnote{{\em I know some of these words!}}:
%   \begin{itemize}
%   \item $\J{P : A \to \U}$ es una fibraci\'on
%   \item $\J{A}$ es el espacio base
%   \item $\J{P(x)}$ es la fibra sobre $\J{x}$
%   \item $\J{\E{x:A}\,P(x)}$ es el espacio total
%   \item $\J{p : x \eq{A} y}$ es un camino en $\J{A}$
%   \item $\J{u : P(x)\ \mapsto\ p^*(u) : P(y)}$
%   \end{itemize}
%   \begin{center}
%   \includegraphics[width=5cm]{fibracion.pdf}
%   \end{center}
%\end{frame}
%
%\begin{frame}[fragile]
%\frametitle{Las familias son fibraciones}
%\begin{itemize}
%\item[] {\bf Lema} ({\em path lifting}){\bf.}
%Sean
%$$\J{P : A \to \U} \text{ una familia sobre $\J{A}$},$$
%$$\J{x : A}, \HS \J{u : P(x)}, \HS \J{p : x \eq{A} y}.$$
%Entonces se tiene:
%$$
%\J{\lift(u, p) : (x, u) = (y, p^*(u))}
%$$
%en $\J{\Sigma_{x:A}\ P(x)}$.
%\end{itemize}
%\end{frame}
%
%\begin{frame}[fragile]
%\frametitle{Las familias son fibraciones}
%\begin{itemize}
%\item[] {\bf Lema.}
%Si $\J{f : \A{x:A}\,P(x)}$, hay una funci\'on
%$$
%  \J{\apd_f : \A{p:x = y}\,p^*(f(x)) \eq{P(y)} f(y)}
%$$
%\end{itemize}
%\end{frame}

\end{document}
